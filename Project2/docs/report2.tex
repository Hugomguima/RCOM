\documentclass[11pt]{article}

\usepackage[english]{babel}
\usepackage{indentfirst}
\usepackage{graphicx}
\usepackage{subfig}
\usepackage[section]{placeins}

\usepackage{xcolor}
\usepackage{listings}
\usepackage{float}
\usepackage{capt-of}

\definecolor{mGreen}{rgb}{0,0.6,0}
\definecolor{mGray}{rgb}{0.5,0.5,0.5}
\definecolor{mPurple}{rgb}{0.58,0,0.82}
\definecolor{backgroundColour}{rgb}{0.95,0.95,0.92}

\lstdefinestyle{CStyle}{
    backgroundcolor=\color{backgroundColour},   
    commentstyle=\color{mGreen},
    keywordstyle=\color{magenta},
    numberstyle=\tiny\color{mGray},
    stringstyle=\color{mPurple},
    basicstyle=\footnotesize,
    breakatwhitespace=false,         
    breaklines=true,                 
    captionpos=b,                    
    keepspaces=true,                 
    numbers=left,                    
    numbersep=5pt,                  
    showspaces=false,                
    showstringspaces=false,
    showtabs=false,                  
    tabsize=2,
    language=C
}


\begin{document}

\begin{titlepage}
	\begin{center}
		\vspace*{1cm}
		
		\Large
		\textbf{Rede de Computadores}
		
		\vspace{0.5cm}
		\large
		Segundo Trabalho Laboratorial
		
		\vspace{1.5cm}
		
		\textbf{Hugo Miguel Monteiro Guimarães}\\
		\textbf{Pedro Varandas da Costa Azevedo da Ponte}
		
		\vspace{5cm}
		
		Trabalho realizado no âmbito da\\
		Unidade Curricular de Redes de Computadores
		
		\vspace{0.8cm}
	
		\includegraphics[width=0.4 \textwidth]{feup_logo.png}
		
		\vspace{1.5cm}		
		
		\large
		Mestrado Integrado em Engenharia Informática e Computação\\
		Faculdade de Engenharia da Universidade do Porto\\
		Porto\\
		23 de dezembro de 2020
	
	\end{center}
\end{titlepage}


\pagebreak
\tableofcontents

\pagebreak


\section*{Sumário}
Este trabalho foi realizado no contexto da cadeira Redes de Computadores, com o objetivo de implementar uma aplicação segundo o protocolo \emph{FTP}, permitindo descarregar um ficheiro a partir de um determinado url.
 
Deste modo, o trabalho foi concluído com sucesso, dado que foi possível implementar uma aplicação que cumprisse os objetivos estabelecidos.


\section{Introdução}
Este trabalho está dividido em 2 partes, sendo a primeira responsável pelo \emph{download} de um ficheiro segundo o protocolo \emph{FTP} e segunda a configuração e análise de uma rede

O relatório pretende descrever detalhadamente a aplicação implementada, estando dividida nas seguintes secções:
\begin{description}
	\item[Parte 1 -] Arquitetura da aplicação de \emph{download} de um ficheiro e os respetivos resultados.
	\item[Parte 2 -] Configuração da rede e análise das seis experiências executadas de acordo com o guião fornecido.
	\item[Conclusões - ] Síntese da informação apresentada nas secções anteriores e reflexão sobre os objetivos de aprendizagem alcançados.
\end{description}


\section{Parte 1 - Aplicação de Download}
A primeira parte deste trabalho consistiu em desenvolver uma aplicação de \emph{download} na linguagem de programação C. Para correr a aplicação, basta compilar os ficheiros usando o \emph{Makefile} e depois correr o comando \textbf{./download ftp://[$<$user$>$:$<$password$>$@]$<$host$>$/$<$url-path$>$}. Esta aplicação permite fazer o download de ficheiros a partir de um servidor do tipo \textbf{FTP}

\subsection{Arquitetura}
Para desenvolver a aplicação, decidimos dividi-la em duas partes: a primeira faz o processamento dos argumentos introduzidos no terminal e que vão ser utilizados para a transferência. A segunda parte é responsável pela conexão ao servidor para que se possa obter o ficheiro desejado.

Para processar os argumentos, desenvolvemos a função \textit{\textbf{parseArguments}}, que divide o argumento nos seus vários componentes: \emph{username, password, host name, path e filename}. Estes componentes vão ser armazenados numa struct \textit{\textbf{arguments}} por nós criada, que vai ser acedida depois para que a transferência do ficheiro ocorra.

Para implementar a segunda parte, primeiramente temos de obter o endereço IP a que o \emph{host} está associado. Para isso, recorremos à função \textit{\textbf{getIP}}, que utiliando maioritariamente código fornecido num exemplo das aulas teórico-práticas, obtém o IP, estando desta forma aptos a iniciar a conexão com o host. De seguida, chama-se a função \textit{\textbf{initConnection}}, que utilizando novamente código fornecido nas aulas, é responsável por ligar o cliente FTP ao servidor através de um socket, utilizando-se a porta 21.

Após a conexão estar estabelecida, envia-se então o primeiro comando "\textbf{USER} 'username'\textbackslash r\textbackslash n" através da função \textit{\textbf{sendData}}, esperando-se pela resposta usando a função \textit{\textbf{receiveAnswer}}. No caso de ser a resposta esperada, envia-se entao o comando "\textbf{PASS} 'password'\textbackslash r\textbackslash n", esperando-se novamente pela resposta do servidor para se poder prosseguir.

As chamadas à função \textit{\textbf{receiveAnswer}} podem retornar essencialmente 3 tipos de respostas diferentes que importa clarificar:

\begin{itemize}
    \item \textbf{Códigos começados por 2 -} Respostas positivas, ou seja, a operação realizada pelo comando enviado foi efetuada com sucesso;
    \item \textbf{Códigod começados por 3 -} Respostas intermédias positivas, ou seja, a operação é válida, mas são necessárias mais informações para que o pedido seja terminado;
    \item \textbf{Códigos começados por 5 -} Respostas negativas, ou seja, ocorreu um erro quando o comando foi enviado.
\end{itemize}

De seguida, são enviados os comandos \textbf{PASV}, que permite entrar no modo passivo e que retorna o endereço IP e a porta para que se possa iniciar um novo \emph{socket} que será responsável pela troca de dados, e "\textbf{RETR} 'path\textunderscore to\textunderscore file\textbackslash r\textbackslash n'", sendo de seguida feito o \emph{download} do ficheiro desejado através da função \textit{\textbf{downloadFile}}. 

Após a transferência estar finalizada, as ligações são fechadas e a aplicação termina.

\subsection{Resultados}

O programa foi testado tanto em modo anónimo como em modo não anónimo, transferindo-se variados ficheiros de diferentes tamanhos e extensões, tendo o programa funcionado corretamente. Caso o ficheiro não exista, seja inválido ou tenha ocorrido algum erro no estabelecimento das ligações ou no envio dos comandos, o programa termina, sendo retornada uma mensagem de erro. 

Ao longo da execução do programa vão sendo imprimidas no terminal diversas mensagens para que seja mais fácil ao utilizadpr acompanhar o desenvolvimento do programa.

\section{Parte 2 - Configuração da Rede e Análise} 

\subsection{Experiência 1 - Configuração de uma rede IP}

\setsecnumdepth{1)} O que são pacotes ARP e qual é a sua finalidade?

\setsecnumdepth{2)} Quais são os endereços IP e MAC dos pacotes ARP e porquê?

\setsecnumdepth{3)} Que pacotes gera o comando \emph{ping}?

\setsecnumdepth{4)} Quais são os endereços MAC e IP dos pacotes \emph{ping}?

\setsecnumdepth{5)} Como determinar se uma trama Ethernet recebido é ARP, IP ou ICMP?

\setsecnumdepth{6)} Como determinar o tamanho de uma trama recebida?

\setsecnumdepth{7)} O que é a interface \emph{loopback} e porque é importante?

\subsection{Experiência 2 - Implementação de 2 VLANS no SWITCH}

\setsecnumdepth{1)} Como configurar a vlany0?

\setsecnumdepth{2)} Quantos domínios \emph{broadcast} existem aqui? Como se pode concluir isso a partir dos registos?

\subsection{Experiência 3 - Configuração de um Router em Linux}

\setsecnumdepth{1)} Que rotas há nos tuxes? Qual o seu significado?

\setsecnumdepth{2)} Que informações contém uma entrada de uma tabela de reencaminhamento?

\setsecnumdepth{3)} Que mensagens ARP e endereços MAC associados são observados e porquê?

\setsecnumdepth{4)} Que pacotes ICMP são observados e porquê?

\setsecnumdepth{5)} Quais são os endereços IP e MAC associados aos pacotes ICMP e porquê?

\subsection{Experiência 4 - Configuração de um Router Comercial e Implementação de NAT}

\setsecnumdepth{1)} Como configurar um \emph{router} estático num \emph{router} comercial?

\setsecnumdepth{2)} Quais são as rotas seguidas pelos pacotes durante a experiência? Explique.

\setsecnumdepth{3)} Como configurar o NAT num \emph{router} comercial?

\setsecnumdepth{4)} O que faz o NAT?

\subsection{Experiência 5 - DNS (Domain Name System)}

\setsecnumdepth{1)} Como configurar o DNS num \emph{host}?

\setsecnumdepth{2)} Que pacotes são trocados pelo DNS e que informações são transportadas? 

\subsection{Experiência 6 - Conexões TCP}

\setsecnumdepth{1)} Quantas conexões TCP são abertas pela aplicação FTP?

\setsecnumdepth{2)} Em que conexão é transportado a informação de controlo do FTP?

\setsecnumdepth{3)} Quais são a fase de uma conexão TCP?

\setsecnumdepth{4)} Como funciona o mecanismo ARQ TCP? Quais são os campos TCP relevantes? Que informação relevante pode ser observada nos logs?

\setsecnumdepth{5)} Como funciona o mecanismo de controlo de congestão do TCP? Quais são os campos relevantes? Como é que o fluxo de dados da conexão evoluiu ao longo do tempo? Está de acordo com o mecanismo de controlo de congestão TCP?

\setsecnumdepth{6)} O fluxo de dados da conexão TCP é afetada pelo aparecimento de uma segunda conexão TCP? Como?

\section{Conclusões}

\section{Anexos}

\end{document}
