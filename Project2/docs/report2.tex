\documentclass[11pt]{article}

\usepackage[english]{babel}
\usepackage{indentfirst}
\usepackage{graphicx}
\usepackage{subfig}
\usepackage[section]{placeins}

\usepackage{xcolor}
\usepackage{listings}
\usepackage{float}
\usepackage{capt-of}

\definecolor{mGreen}{rgb}{0,0.6,0}
\definecolor{mGray}{rgb}{0.5,0.5,0.5}
\definecolor{mPurple}{rgb}{0.58,0,0.82}
\definecolor{backgroundColour}{rgb}{0.95,0.95,0.92}

\lstdefinestyle{CStyle}{
    backgroundcolor=\color{backgroundColour},   
    commentstyle=\color{mGreen},
    keywordstyle=\color{magenta},
    numberstyle=\tiny\color{mGray},
    stringstyle=\color{mPurple},
    basicstyle=\footnotesize,
    breakatwhitespace=false,         
    breaklines=true,                 
    captionpos=b,                    
    keepspaces=true,                 
    numbers=left,                    
    numbersep=5pt,                  
    showspaces=false,                
    showstringspaces=false,
    showtabs=false,                  
    tabsize=2,
    language=C
}


\begin{document}

\begin{titlepage}
	\begin{center}
		\vspace*{1cm}
		
		\Large
		\textbf{Protocolo de Ligação de Dados}
		
		\vspace{0.5cm}
		\large
		Primeiro Trabalho Laboratorial
		
		\vspace{1.5cm}
		
		\textbf{Hugo Miguel Monteiro Guimarães}\\
		\textbf{Pedro Varandas da Costa Azevedo da Ponte}
		
		\vspace{5cm}
		
		Trabalho realizado no âmbito da\\
		Unidade Curricular de Redes de Computadores
		
		\vspace{0.8cm}
	
		Colocar imagem do logo aqui
		
		\vspace{1.5cm}		
		
		\large
		Mestrado Integrado em Engenharia Informática e Computação\\
		Faculdade de Engenharia da Universidade do Porto\\
		Porto\\
		17 de novembro de 2020
	
	\end{center}
\end{titlepage}


\pagebreak
\tableofcontents

\pagebreak


\section*{Sumário}
Este trabalho foi realizado no contexto da cadeira Redes de Computadores, com o objetivo de implementar uma aplicação segundo o protocolo \emph{FTP}, permitindo descarregar um ficheiro a partir de um determinado url.
 
Deste modo, o trabalho foi concluído com sucesso, dado que foi possível implementar uma aplicação que cumprisse os objetivos estabelecidos.


\section{Introdução}
Este trabalho está dividido em 2 partes, sendo a primeira responsável pelo \emph{download} de um ficheiro segundo o protocolo \emph{FTP} e segunda a configuração e análise de uma rede

O relatório pretende descrever detalhadamente a aplicação implementada, estando dividida nas seguintes secções:
\begin{description}
	\item[Parte 1 -] Arquitetura da aplicação de \emph{download} de um ficheiro e os respetivos resultados.
	\item[Parte 2 -] Configuração da rede e análise das seis experiências executadas de acordo com o guião fornecido.
	\item[Conclusões - ] Síntese da informação apresentada nas secções anteriores e reflexão sobre os objetivos de aprendizagem alcançados.
\end{description}


\section{Parte 1 - Aplicação de Download}

\subsection{Arquitetura}

\subsection{Resultados}


\section{Parte 2 - Configuração da Rede e Análise} 

\subsection{Experiência 1 - Configuração de uma rede IP}

\subsection{Experiência 2 - Implementação de 2 VLANS no SWITCH}

\subsection{Experiência 3 - Configuração de um Router em Linux}

\subsection{Experiência 4 - Configuração de um Router Comercial e Implementação de NAT}

\subsection{Experiência 5 - DNS (Domain Name System)}

\subsection{Experiência 6 - Conexões TCP}

\section{Conclusões}

\section{Anexos}

\end{document}
